\documentclass[10pt,a4paper]{article}
\usepackage[utf8]{inputenc}
\usepackage{amsmath}
\usepackage{amsfonts}
\usepackage{amssymb}
\author{Qin ZHOU \\
    fatestudio@gmail.com
    \and
    Liang XIA \\
    liangxia2006@gmail.com
    }
\title{CS240A Proposal: Parallelizing Elliptic Curve Cryptography}
\begin{document}
\maketitle
\section{Introduction}
\indent Elliptic Curve Cryptography (ECC) is a public-key technology
that offers performance advantages at
   higher security levels.  It includes an elliptic curve version of the
   Diffie-Hellman key exchange protocol \cite{DH1976} and elliptic curve
   versions of the ElGamal Signature Algorithm \cite{E1985}.  The adoption of
   ECC has been slower than had been anticipated, perhaps due to the
   lack of freely available normative documents and uncertainty over
   intellectual property rights. \cite{RFC6090} \\ \\
Elliptic Curve Cryptography can be parallelized in several ways: \\ \\
First we can parallelize the computation of $2P$ in Elliptic Curve, which means parallelizing the computation of equation:
\begin{equation}
x_3 \equiv m^2 - 2x_1\ (mod\ p) \\
\end{equation}
\begin{equation}
y_3 \equiv m(x_1 - x_3) - y_1\ (mod\ p) \\
\end{equation}
\begin{equation}
m \equiv \frac{3x_1^2 + A}{2y_1}\ (mod\ p)
\end{equation}
Second we can parallelize several attacks towards ECC. For example, Pollard's $\lambda$ method can be parallized by using several different random starting points \cite{ECC Book}. There are several other existing parallel attacks, and we can also develop our new attacking methods. \\
\section{Schedule}
1. Read some materials; \\
2. Implement an Elliptic Curve Cryptography algorithm (basic ECC components, or a complete ECDHE) on Triton.\\
3. Try to use OpenMP, MPI, and pthread to efficiently parallelize
the ECC algorithm ; \\
4. Evaluate the performance data, such as MFLOPS, parallel time, and speedup compared to the original ECC implementation.\\
5. Implement several parallized attacks towards ECC on Triton.\\
\begin{thebibliography}{}
    \bibitem{ECC Book} L. C. Washington, "Elliptic Curves Number Theory and Cryptography", Second Edition, 2008
    \bibitem{RFC6090} D. McGrew, K. Igoe and M. Salter, "Fundamental Elliptic Curve Cryptography Algorithms", RFC6090
    \bibitem{DH1976} Diffie, W. and M. Hellman, "New Directions in
                Cryptography", IEEE Transactions in Information
                Theory IT-22, pp. 644-654, 1976.

    \bibitem{E1985} ElGamal, T., "A public key cryptosystem and a signature
                scheme based on discrete logarithms", IEEE Transactions
                on Information Theory, Vol. 30, No. 4, pp. 469-472,
                1985.

    \end{thebibliography}
\end{document}
